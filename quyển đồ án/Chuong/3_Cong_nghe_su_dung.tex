\documentclass[../DoAn.tex]{subfiles}
\begin{document}

% Chương này có độ dài không quá 10 trang. Nếu cần trình bày dài hơn, sinh viên đưa vào phần phụ lục. Chú ý đây là kiến thức đã có sẵn; SV sau khi tìm hiểu được thì phân tích và tóm tắt lại. Sinh viên không trình bày dài dòng, chi tiết. 

% Với đồ án ứng dụng, sinh viên để tên chương là “Công nghệ sử dụng”. Trong chương này, sinh viên giới thiệu về các công nghệ, nền tảng sử dụng trong đồ án. Sinh viên cũng có thể trình bày thêm nền tảng lý thuyết nào đó nếu cần dùng tới.

% Với đồ án nghiên cứu, sinh viên đổi tên chương thành “Cơ sở lý thuyết”. Khi đó, nội dung cần trình bày bao gồm: Kiến thức nền tảng, cơ sở lý thuyết, các thuật toán, phương pháp nghiên cứu, v.v.

% Với từng công nghệ/nền tảng/lý thuyết được trình bày, sinh viên phải phân tích rõ công nghệ/nền tảng/lý thuyết đó dùng để để giải quyết vấn đề/yêu cầu cụ thể nào ở Chương 2. Hơn nữa, với từng vấn đề/yêu cầu, sinh viên phải liệt kê danh sách các công nghệ/hướng tiếp cận tương tự có thể dùng làm lựa chọn thay thế, rồi giải thích rõ sự lựa chọn của mình.

% Lưu ý: Nội dung ĐATN phải có tính chất liên kết, liền mạch, và nhất quán. Vì vậy, các công nghệ/thuật toán trình bày trong chương này phải khớp với nội dung giới thiệu của sinh viên ở phần trước đó. 

% Trong chương này, để tăng tính khoa học và độ tin cậy, sinh viên nên chỉ rõ nguồn kiến thức mình thu thập được ở tài liệu nào, đồng thời đưa tài liệu đó vào trong danh sách tài liệu tham khảo rồi tạo các tham chiếu chéo (xem hướng dẫn ở phụ lục A.7).

\section{Java Springboot}
\label{section:3.1}
\subsection{Tổng quan}
\label{subsection:3.1.1}
Spring Boot \cite{springboot} là một framework mã nguồn mở được xây dựng trên nền tảng của Spring Framework, được thiết kế để đơn giản hóa việc phát triển các ứng dụng Java. Với Spring Boot, các nhà phát triển có thể dễ dàng tạo ra các ứng dụng standalone hoặc các microservices một cách nhanh chóng và hiệu quả.

Một trong những điểm nổi bật của Spring Boot là khả năng tự động cấu hình (auto-configuration), giúp giảm bớt cấu hình phức tạp mà người dùng thường phải thực hiện khi sử dụng Spring Framework truyền thống. Spring Boot cung cấp các mẫu cấu hình mặc định cho nhiều tình huống khác nhau, nhưng vẫn cho phép tùy chỉnh khi cần thiết.

Spring Boot cũng tích hợp sẵn các công cụ và tính năng hữu ích như Spring Boot Actuator, giúp theo dõi và quản lý các ứng dụng đang chạy, và Spring Boot CLI, cho phép các nhà phát triển viết các ứng dụng Spring bằng cách sử dụng cú pháp ngắn gọn và dễ hiểu.

Ngoài ra, Spring Boot hỗ trợ tốt cho việc phát triển các ứng dụng microservices,các ứng dụng theo mô hình monolytic, nhờ vào các công cụ và thư viện tích hợp như Spring Cloud. Điều này giúp dễ dàng xây dựng các hệ thống phân tán với khả năng mở rộng và tính chịu lỗi cao.

Với Spring Boot, các nhà phát triển có thể tập trung vào việc viết mã logic của ứng dụng, trong khi framework sẽ lo liệu phần lớn các chi tiết cấu hình và thiết lập, giúp tăng năng suất và giảm thiểu sai sót.
\subsection{Triển khai trong đồ án}
\label{subsection:3.1.2}
Trong dự án này, Spring Boot được triển khai để làm nền tảng server-side, giúp xử lý các yêu cầu từ client, quản lý dữ liệu và cung cấp các api cho các thành phần khác của hệ thống, xử lý các yêu cầu HTTP, định tuyến và, quản lý logic ứng dụng.

\section{AngularJS}
\label{section:3.2}
\subsection{Tỏng quan}
\label{subsection:3.2.1}
Angular \cite{jain2014angularjs} là một framework mã nguồn mở được phát triển bởi Google, dùng để xây dựng các ứng dụng web động và phức tạp. Được viết bằng TypeScript, Angular giúp các nhà phát triển tạo ra các ứng dụng web một cách hiệu quả và dễ dàng hơn nhờ vào các tính năng mạnh mẽ và các công cụ hỗ trợ.

Một trong những đặc điểm nổi bật của Angular là khả năng tạo ra các ứng dụng đơn trang (Single Page Application - SPA), nơi mà mọi tương tác của người dùng đều được xử lý một cách nhanh chóng và mượt mà mà không cần phải tải lại toàn bộ trang. Angular đạt được điều này thông qua việc sử dụng các thành phần (components) để tổ chức và tái sử dụng mã nguồn.

Angular cung cấp nhiều công cụ và thư viện tích hợp sẵn như Angular CLI, giúp tạo mới, phát triển, kiểm tra và triển khai các ứng dụng một cách dễ dàng. Angular cũng có hệ thống dữ liệu mạnh mẽ thông qua Angular Services, cho phép quản lý trạng thái và giao tiếp với các API một cách hiệu quả.

Framework này cũng hỗ trợ Dependency Injection, giúp quản lý và sử dụng các dịch vụ một cách linh hoạt và hiệu quả. Hơn nữa, Angular có một cộng đồng lớn và sôi động, cung cấp nhiều tài liệu, plugin và thư viện bổ trợ, giúp các nhà phát triển dễ dàng tìm kiếm sự hỗ trợ và mở rộng tính năng cho ứng dụng của mình.

Với khả năng mạnh mẽ và linh hoạt, Angular là một lựa chọn phổ biến cho việc phát triển các ứng dụng web hiện đại, từ các ứng dụng nhỏ đến các hệ thống lớn và phức tạp.

\subsection{Triển khai trong đồ án}
\label{subsection:3.2.2}
Trong dự án này, Angular được triển khai để tạo giao diện người dùng, xây diện giao diên theo hướng SPA, gửi các request HTTP về bên phía server, nhận các response từ bên server trả về.

\section{MySQL}
\label{section:3.3}
\subsection{Tỏng quan}
\label{subsection:3.3.1}
MySQL là một hệ quản trị cơ sở dữ liệu quan hệ (Relational Database Management System - RDBMS) mã nguồn mở, được phát triển và phân phối bởi Oracle Corporation. Đây là một trong những hệ quản trị cơ sở dữ liệu phổ biến nhất trên thế giới, được ưa chuộng nhờ khả năng xử lý truy vấn nhanh chóng và hiệu quả, ngay cả khi làm việc với các cơ sở dữ liệu lớn. MySQL hỗ trợ nhiều tính năng mạnh mẽ, bao gồm bảo mật cao, khả năng mở rộng linh hoạt và hỗ trợ đa nền tảng, cho phép nó chạy trên các hệ điều hành phổ biến như Windows, Linux và MacOS. Là phần mềm mã nguồn mở, MySQL không chỉ miễn phí mà còn cho phép người dùng tùy chỉnh và cải tiến mã nguồn theo nhu cầu riêng, tạo điều kiện cho sự phát triển cộng đồng mạnh mẽ.

MySQL nổi bật với hiệu suất cao, được thiết kế để xử lý các truy vấn cơ sở dữ liệu một cách nhanh chóng và hiệu quả, đồng thời hỗ trợ các chỉ mục và tối ưu hóa truy vấn để cải thiện hiệu suất làm việc. Khả năng mở rộng của MySQL cũng rất ấn tượng, với các phương pháp phân vùng, sao chép và các kỹ thuật khác giúp quản lý cơ sở dữ liệu lớn một cách hiệu quả. Về bảo mật, MySQL cung cấp nhiều tính năng như xác thực người dùng, kiểm soát truy cập, mã hóa dữ liệu và hỗ trợ SSL để bảo vệ dữ liệu trong quá trình truyền tải. Hơn nữa, MySQL dễ sử dụng với ngôn ngữ truy vấn chuẩn (SQL) dễ học và nhiều công cụ giao diện người dùng đồ họa (GUI) như MySQL Workbench, giúp việc quản lý cơ sở dữ liệu trở nên dễ dàng hơn.

Ứng dụng của MySQL rất đa dạng, từ phát triển web đến doanh nghiệp và phân tích dữ liệu. Trong phát triển web, MySQL là lựa chọn phổ biến cho các ứng dụng web, đặc biệt là những ứng dụng được xây dựng bằng các ngôn ngữ như PHP, Python và Ruby. Các hệ thống quản lý nội dung (CMS) như WordPress, Joomla và Drupal đều sử dụng MySQL. Trong lĩnh vực doanh nghiệp và thương mại điện tử, MySQL được sử dụng để quản lý dữ liệu khách hàng, đơn hàng và các thông tin liên quan khác, với các nền tảng thương mại điện tử như Magento và Shopify sử dụng MySQL làm cơ sở dữ liệu. MySQL cũng được sử dụng trong phân tích dữ liệu nhờ khả năng xử lý các truy vấn phức tạp và tích hợp với các công cụ phân tích dữ liệu như Apache Hadoop và Tableau. Trong phát triển ứng dụng di động, MySQL có thể hoạt động như một backend cho các ứng dụng cần lưu trữ và truy xuất dữ liệu.

Với sự hỗ trợ liên tục từ Oracle và cộng đồng mã nguồn mở, MySQL không ngừng phát triển và cải thiện. Các phiên bản mới thường xuyên được phát hành với các tính năng và tối ưu hóa mới, giúp MySQL duy trì vị trí là một trong những hệ quản trị cơ sở dữ liệu phổ biến và mạnh mẽ nhất. Trong tương lai, MySQL dự kiến sẽ tiếp tục là lựa chọn hàng đầu cho các nhà phát triển và doanh nghiệp, nhờ vào sự linh hoạt, hiệu suất cao và tính năng bảo mật mạnh mẽ, đáp ứng tốt các nhu cầu ngày càng phức tạp của ứng dụng hiện đại.

\subsection{Triển khai trong đồ án}
\label{subsection:3.3.2}
Trong dự án này, MySQL được dùng để lưu trữ cơ sở dữ liệu, giao tiếp với server, thực hiện các câu truy vấn đề lấy dữ liệu, thay đổi dữ liệu.


\section{VNPAY}
\label{section:3.4}
\subsection{Tổng quan}
VNPAY, một thương hiệu nổi bật trong lĩnh vực công nghệ tài chính (FinTech) tại Việt Nam, đã không ngừng phát triển và khẳng định vị thế của mình trên thị trường thanh toán điện tử. Thành lập vào năm 2007, VNPAY đã nhanh chóng trở thành một trong những công ty hàng đầu trong việc cung cấp các giải pháp thanh toán điện tử và dịch vụ tài chính số tại Việt Nam. Với tầm nhìn tiên phong và chiến lược phát triển bền vững, VNPAY đã và đang góp phần quan trọng vào việc thúc đẩy sự phát triển của nền kinh tế số tại Việt Nam.

Một trong những sản phẩm nổi bật nhất của VNPAY chính là cổng thanh toán VNPAY. Đây là một nền tảng thanh toán điện tử hiện đại và toàn diện, cho phép các doanh nghiệp và người tiêu dùng thực hiện các giao dịch trực tuyến một cách nhanh chóng, an toàn và tiện lợi. Cổng thanh toán VNPAY được xây dựng dựa trên các công nghệ tiên tiến và hiện đại nhất, nhằm đảm bảo tính bảo mật cao và khả năng xử lý giao dịch nhanh chóng.

Về mặt công nghệ, cổng thanh toán VNPAY áp dụng các chuẩn bảo mật quốc tế như PCI DSS (Payment Card Industry Data Security Standard) để bảo vệ thông tin thẻ và dữ liệu khách hàng. Hệ thống này còn tích hợp các công nghệ mã hóa (encryption) và xác thực hai yếu tố (2FA - Two-Factor Authentication) để tăng cường tính an toàn trong quá trình giao dịch. Việc sử dụng các công nghệ này không chỉ giúp bảo vệ thông tin cá nhân của người dùng mà còn ngăn chặn các hành vi gian lận và tấn công mạng.

Cổng thanh toán VNPAY cung cấp một loạt các tính năng đa dạng và hữu ích, đáp ứng nhu cầu của cả doanh nghiệp và người tiêu dùng. Một trong những tính năng nổi bật là khả năng hỗ trợ nhiều phương thức thanh toán khác nhau. Người dùng có thể thanh toán qua thẻ tín dụng, thẻ ghi nợ, ví điện tử, mã QR, và thậm chí là các phương thức thanh toán truyền thống như chuyển khoản ngân hàng. Điều này mang lại sự linh hoạt và tiện lợi cho người dùng, đồng thời mở rộng khả năng chấp nhận thanh toán cho các doanh nghiệp.

VNPAY QR, một phần của cổng thanh toán VNPAY, là một giải pháp thanh toán qua mã QR độc đáo và tiện lợi. Với VNPAY QR, người dùng chỉ cần sử dụng ứng dụng mobile banking hoặc ví điện tử để quét mã QR và thực hiện thanh toán một cách nhanh chóng. Tính năng này không chỉ giúp người dùng tiết kiệm thời gian mà còn giảm thiểu rủi ro về bảo mật khi không cần phải nhập thông tin thẻ hay tài khoản ngân hàng trực tiếp.

Ngoài ra, cổng thanh toán VNPAY còn hỗ trợ các tính năng quản lý giao dịch và báo cáo tài chính chi tiết. Doanh nghiệp có thể theo dõi và quản lý các giao dịch một cách dễ dàng thông qua giao diện quản trị thân thiện và trực quan. Hệ thống báo cáo tài chính của VNPAY cung cấp các thông tin chi tiết về doanh thu, số lượng giao dịch, và các chỉ số tài chính khác, giúp doanh nghiệp đưa ra các quyết định kinh doanh kịp thời và chính xác.

VNPAY cũng đặc biệt chú trọng đến trải nghiệm người dùng. Giao diện của cổng thanh toán VNPAY được thiết kế đơn giản, dễ sử dụng và thân thiện với người dùng. Quá trình thanh toán được tối ưu hóa để giảm thiểu số bước cần thiết, giúp người dùng hoàn tất giao dịch một cách nhanh chóng và thuận tiện.

Không chỉ dừng lại ở việc cung cấp các dịch vụ thanh toán, VNPAY còn tích cực hợp tác với các ngân hàng, tổ chức tài chính và các đối tác chiến lược khác để mở rộng hệ sinh thái thanh toán điện tử. Điều này không chỉ giúp VNPAY nâng cao chất lượng dịch vụ mà còn tạo ra nhiều giá trị gia tăng cho khách hàng.
\subsection{Triển khai trong đồ án}
\label{subsection:3.4.2}
Trong đồ án này, hệ thống cổng thanh toán được dùng để xử lý thanh toán trực tuyến thông qua thẻ, do đây là đồ án cá nhân, không có mã số doanh nghiệp nên mọi giao dịch đều được thực hiện trên môi trường test.

\section{JSON Web Token}
\label{section 3.5}
\subsection{Tổng quan}
\label{subsection:3.5.1}
JSON Web Token (JWT) là một tiêu chuẩn mã nguồn mở (RFC 7519) định dạng token, được sử dụng để đại diện cho các claims (yêu cầu) được chuyển giữa hai bên. Những claims này thường là thông tin về người dùng hoặc các quyền truy cập, và chúng được mã hóa dưới dạng JSON. JWT là một công nghệ phổ biến trong việc xác thực và ủy quyền, đặc biệt là trong các ứng dụng web và dịch vụ RESTful.

JWT được tạo ra với ba phần chính: Header, Payload và Signature. Phần Header thường chứa hai phần: loại token, ở đây là JWT, và thuật toán ký (ví dụ: HMAC SHA256 hoặc RSA). Payload chứa các claims, đó là dữ liệu mà chúng ta muốn truyền tải. Claims có thể là thông tin xác thực của người dùng (như ID người dùng) hoặc các thông tin khác mà hai bên muốn truyền tải an toàn. Phần cuối cùng là Signature, được tạo ra bằng cách kết hợp mã hóa phần Header và Payload bằng một khóa bí mật hoặc khóa riêng (trong trường hợp sử dụng các thuật toán không đối xứng như RSA). Kết quả là một chuỗi mã hóa duy nhất có thể được sử dụng để xác minh tính toàn vẹn của thông tin và xác nhận danh tính của người gửi.

Một trong những lợi ích chính của JWT là tính tự chứa (self-contained), nghĩa là tất cả thông tin cần thiết để xác thực và ủy quyền đều được chứa trong chính token. Điều này loại bỏ nhu cầu lưu trữ trạng thái người dùng trên server, giúp tăng hiệu quả và khả năng mở rộng của ứng dụng. Khi một người dùng đăng nhập, server sẽ tạo ra một JWT và trả lại cho người dùng. Người dùng sau đó sẽ gửi token này trong mỗi yêu cầu đến server (thường là trong header của HTTP request), và server sẽ xác thực token mà không cần phải truy vấn cơ sở dữ liệu mỗi lần.

JWT được sử dụng rộng rãi trong các hệ thống xác thực dựa trên token. Một trường hợp sử dụng phổ biến là trong các ứng dụng web hiện đại, nơi JWT giúp thực hiện các thao tác đăng nhập một lần (single sign-on - SSO). Khi một người dùng đăng nhập vào một ứng dụng, JWT sẽ được tạo ra và sử dụng để truy cập vào nhiều dịch vụ khác nhau mà không cần phải đăng nhập lại. Điều này mang lại trải nghiệm người dùng mượt mà hơn và giảm thiểu các vấn đề về quản lý phiên làm việc (session management).

Một ứng dụng khác của JWT là trong việc ủy quyền (authorization). Ví dụ, một API có thể sử dụng JWT để đảm bảo rằng chỉ những yêu cầu có token hợp lệ mới được phép truy cập. JWT cũng có thể chứa thông tin về quyền hạn của người dùng, giúp server quyết định xem người dùng có quyền thực hiện một hành động cụ thể hay không. Điều này tăng cường bảo mật và kiểm soát truy cập trong các ứng dụng.

Mặc dù JWT mang lại nhiều lợi ích, nhưng cũng cần chú ý đến các vấn đề bảo mật. Vì JWT thường được truyền qua Internet, nên việc bảo vệ thông tin nhạy cảm là rất quan trọng. Một số biện pháp bảo mật bao gồm sử dụng HTTPS để mã hóa giao tiếp, thiết lập thời gian hết hạn cho token, và sử dụng các thuật toán ký mạnh để đảm bảo tính toàn vẹn và xác thực của token. Ngoài ra, việc lưu trữ khóa bí mật (secret key) phải được thực hiện cẩn thận để tránh việc bị lộ, dẫn đến nguy cơ bị giả mạo token.

\subsection{Triển khai trong đồ án}
\label{subsection:3.5.2}
Trong đồ án, JWT được sử dụng để xác thực người dùng, vai trò và xác thực các API được sủ dụng trong hệ thống.

\section{Bcrypt}
\label{section:3.6}
\subsection{Tổng quan}
\label{subsection:3.6.1}
Bcrypt là một thuật toán mã hóa mật khẩu được thiết kế để chống lại các cuộc tấn công bạo lực và từ điển. Nó được phát triển vào năm 1999 bởi Niels Provos và David Mazières. Bcrypt là một cải tiến của thuật toán cổ điển Blowfish, được thiết kế để mã hóa mật khẩu một cách an toàn và hiệu quả.

Một trong những đặc điểm chính của Bcrypt là khả năng thích ứng với sức mạnh tính toán ngày càng tăng. Thuật toán này sử dụng một số vòng lặp để tăng độ phức tạp của quá trình mã hóa, giúp bảo vệ mật khẩu khỏi các cuộc tấn công bạo lực. Số lượng vòng lặp có thể được điều chỉnh để phù hợp với sức mạnh tính toán hiện tại, đảm bảo mật khẩu vẫn an toàn ngay cả khi phần cứng trở nên mạnh hơn.

Bcrypt cũng có một số lợi ích khác so với các thuật toán mã hóa mật khẩu truyền thống như MD5 và SHA-1. Đầu tiên, nó sử dụng một phương pháp mã hóa dựa trên sự thay đổi của "salt" - một chuỗi ngẫu nhiên được thêm vào mật khẩu trước khi mã hóa. Điều này làm cho mỗi bản mã hóa mật khẩu trở nên duy nhất, ngăn chặn các cuộc tấn công dựa trên bảng tra cứu. Thứ hai, Bcrypt sử dụng một số lượng bộ nhớ lớn hơn so với các thuật toán khác, làm chậm quá trình mã hóa và khiến các cuộc tấn công bạo lực trở nên khó khăn hơn.

Bcrypt được sử dụng rộng rãi trong nhiều ứng dụng và hệ thống, bao gồm cả các ứng dụng web, ứng dụng di động và các hệ thống xác thực khác. Nó được coi là một giải pháp mã hóa mật khẩu an toàn và đáng tin cậy, đặc biệt là trong các ứng dụng yêu cầu bảo mật cao.

\subsection{Triển khai}
\label{subsection:3.6.2}
Trong đồ án, Bcrypt được sử dụng để mã hóa cho mật khẩu rồi lưu vào cơ sở dữ liệu.

\section{Bootstrap 5}
\label{section:3.7}
\subsection{Tổng quan}
\label{subsection:3.7.1}
Bootstrap 5 là một trong những framework phổ biến nhất để phát triển giao diện web responsive và mobile-first. Được ra mắt vào năm 2011, Bootstrap đã trở thành một trong những công cụ thiết kế web phổ biến nhất, cung cấp một tập hợp các thành phần UI và các công cụ hữu ích khác.

Phiên bản mới nhất, Bootstrap 5, được giới thiệu vào năm 2021 với nhiều cải tiến đáng chú ý. Một trong những thay đổi lớn nhất là sự ra đi của jQuery, trước đây là một phần không thể thiếu của Bootstrap. Thay vào đó, Bootstrap 5 sử dụng Vanilla JavaScript, giúp giảm kích thước tệp và tăng tốc độ tải trang.

Ngoài ra, Bootstrap 5 cũng đem lại nhiều cải tiến về giao diện người dùng. Các thành phần như nút, form, navbar và card được thiết kế lại với giao diện sạch sẽ và hiện đại hơn. Hệ thống grid cũng được cải thiện, cho phép tạo bố cục phức tạp hơn một cách dễ dàng.

Một trong những tính năng nổi bật nhất của Bootstrap 5 là khả năng tùy biến. Các biến CSS và Sass đã được cải thiện, cho phép nhà phát triển tùy chỉnh giao diện một cách linh hoạt hơn. Ngoài ra, Bootstrap 5 cũng giới thiệu các tiện ích mới như Offcanvas, Scrollspy và Toasts.

\subsection{Triển khai trong đồ án}
\label{subsection:3.7.2}
Trong đồ án, Bootstrap 5 được sử dụng để tạo giao diện đẹp hơn, phù hợp với xu hướng hiện nay.

\section{Cloudinary}
\label{section:3.8}
\subsection{Tổng quan}
\label{subsection:3.8.1}
Cloudinary là một nền tảng quản lý nội dung đa phương tiện dựa trên đám mây, cung cấp một bộ giải pháp tích hợp và đầy đủ chức năng để xử lý, lưu trữ, tối ưu hóa và phân phối nội dung kỹ thuật số. Được xây dựng trên nền tảng công nghệ hiện đại và kiến trúc phân tán, Cloudinary giúp các doanh nghiệp và nhà phát triển giải phóng khỏi những rào cản về hạ tầng, nguồn lực và thời gian, tập trung vào việc tạo ra trải nghiệm người dùng tuyệt vời thông qua nội dung đa phương tiện.

Về mặt công nghệ, Cloudinary được xây dựng dựa trên các nguyên tắc và kiến trúc hiện đại, đảm bảo tính linh hoạt, khả năng mở rộng và khả năng chịu lỗi cao. Nền tảng này sử dụng các công nghệ đám mây tiên tiến như Amazon Web Services (AWS), Google Cloud Platform (GCP) và Microsoft Azure để cung cấp khả năng lưu trữ, xử lý và phân phối nội dung với hiệu suất và tính sẵn sàng cao. Cloudinary còn khai thác các công nghệ máy học và trí tuệ nhân tạo để tối ưu hóa và tự động hóa các quy trình như nhận dạng, gắn thẻ và phân loại nội dung.

Một trong những tính năng nổi bật của Cloudinary là khả năng xử lý và tối ưu hóa hình ảnh và video. Nền tảng này cung cấp các công cụ tiên tiến để thao tác, chỉnh sửa, nén và chuyển đổi các tệp đa phương tiện, đồng thời tự động áp dụng các kỹ thuật tối ưu hóa như lazy loading, adaptive bitrate streaming và phân phối nội dung thông qua mạng phân phối nội dung (CDN). Điều này giúp giảm tải trọng trang web, cải thiện tốc độ tải và tăng cường trải nghiệm người dùng.

Cloudinary cũng cung cấp các công cụ để quản lý và phân phối nội dung một cách hiệu quả. Nền tảng này cho phép các doanh nghiệp và nhà phát triển lưu trữ, tổ chức và tìm kiếm nội dung một cách dễ dàng thông qua các tính năng như thư viện phương tiện, phân loại thông minh và tìm kiếm nâng cao. Các tính năng này giúp giảm thiểu thời gian và nỗ lực cần thiết để quản lý và truy xuất các tài nguyên đa phương tiện.

Một điểm nhấn khác của Cloudinary là khả năng tích hợp linh hoạt với các nền tảng và ứng dụng khác. Nền tảng này cung cấp các API, SDK và công cụ tích hợp để dễ dàng kết nối với các hệ thống hiện có như hệ thống quản lý nội dung (CMS), hệ thống thương mại điện tử và các ứng dụng web/di động khác. Điều này giúp các doanh nghiệp và nhà phát triển tích hợp các tính năng quản lý và phân phối nội dung đa phương tiện một cách nhachóng vánh vào các sản phẩm, dịch vụ của họ.

Ngoài ra, Cloudinary cũng chú trọng vào các khía cạnh như an toàn, bảo mật và tuân thủ các tiêu chuẩn ngành. Nền tảng này đáp ứng các yêu cầu an toàn và bảo mật nghiêm ngặt, bao gồm mã hóa dữ liệu, xác thực và ủy quyền người dùng, cũng như tuân thủ các tiêu chuẩn như GDPR, HIPAA và PCI DSS.

\subsection{Triển khai trong đồ án}
\label{subsection3.8.2}
Trong đồ án này, Cloudinary được sử dụng để triển khai lưu trữ ảnh cho các cuốn sách.
\end{document}