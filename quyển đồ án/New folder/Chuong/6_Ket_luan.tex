\documentclass[../DoAn.tex]{subfiles}
\begin{document}
\section{Kết luận}
\label{section:6.1}
% Sinh viên so sánh kết quả nghiên cứu hoặc sản phẩm của mình với các nghiên cứu hoặc sản phẩm tương tự.

% Sinh viên phân tích trong suốt quá trình thực hiện ĐATN, mình đã làm được gì, chưa làm được gì, các đóng góp nổi bật là gì, và tổng hợp những bài học kinh nghiệm rút ra nếu có.

Trong quá trình thực hiện đề tài, hệ thống bán sách trực tuyến đã được nghiên cứu và phát triển nhằm giải quyết các hạn chế của hệ thống bán lẻ sách truyền thống. Hệ thống mà tôi phát triển có nhiều điểm tương đồng với các nền tảng thương mại điện tử như Fahasa,VinaBook, tuy nhiên nổi bật với khả nang triển khai cho các nhà sách có quy mô vừa vả nhỏ với các tính năng cơ bản như quản lý danh mục sách, xử lý đơn hàng, và theo dõi tình trạng giao hàng.

Trải qua quá trình thực hiện, tôi đã đạt được nhiều thành tựu quan trọng. Hệ thống đã được thiết kế và triển khai với các tính năng hoàn chỉnh như tìm kiếm và mua sách trực tuyến, xử lý đơn hàng, đồng thời cung cấp giao diện thân thiện và dễ sử dụng. Tuy nhiên, vẫn còn một số hạn chế cần được khắc phục.

Đầu tiên, hệ thống chưa triển khai đầy đủ các tính năng nâng cao như chức năng đánh giá và bình luận sách, làm giảm khả năng tương tác và trao đổi ý kiến giữa các khách hàng về sản phẩm. Hệ thống cũng chưa hỗ trợ việc quản lý nhiều nhà cung cấp cùng một lúc, gây khó khăn trong việc đa dạng hóa nguồn hàng và tăng cường sự cạnh tranh giữa các nhà cung cấp.

Thứ hai, vẫn tồn tại hạn chế các chức năng cá nhân hóa dành cho người dùng. Ví dụ như chức năng gợi ý tìm kiếm dựa theo các kết quả tìm kiếm gần đẩy của người dùng, hệ gợi ý sách dựa trên sở thích của người dùng.

Những hạn chế này là những bài học quan trọng để tôi tiếp tục hoàn thiện hệ thống, nâng cao chất lượng và hiệu suất của nó trong tương lai. Qua quá trình này, tôi đã học được nhiều kinh nghiệm và kiến thức quý báu, từ quản lý dự án đến áp dụng các công nghệ web hiện đại như AngularJS, NodeJS, MySQL,Spring Boot, VNPAY và API Cloudinary. Kỹ năng giải quyết vấn đề và xử lý các tình huống phát sinh cũng đã được cải thiện đáng kể.

Tóm lại, hệ thống bán sách trực tuyến đã đáp ứng được mục tiêu đề ra, mang lại giải pháp hiệu quả cho nhu cầu mua sắm sách trực tuyến trong xã hội ngày nay. Mặc dù vẫn còn những hạn chế nhất định, nhưng những đóng góp và bài học kinh nghiệm này sẽ là nền tảng vững chắc cho các dự án tương lai của tôi.

\section{Hướng phát triển}
% Trong phần này, sinh viên trình bày định hướng công việc trong tương lai để hoàn thiện sản phẩm hoặc nghiên cứu của mình.

% Trước tiên, sinh viên trình bày các công việc cần thiết để hoàn thiện các chức năng/nhiệm vụ đã làm. Sau đó sinh viên phân tích các hướng đi mới cho phép cải thiện và nâng cấp các chức năng/nhiệm vụ đã làm.

Như đã đề cập trong phần \ref{section:6.1} để hoàn thiện hệ thống bán sách trực tuyến mà tôi đã nghiên cứu và phát triển, tôi đặt ra một số hướng phát triển trong tương lai như sau:
\begin{enumerate}
    \item[(i)] \textbf{Thêm các chức năng nâng cao}: Tích hợp các chức năng nâng cao hơn nhằm tăng tương tác cho người dùng như: Chức năng bình luận, đánh giá sách, thêm vào mục "yêu thích", chức năng chat trực tuyến với admin. Đây đều là những tính năng nâng cao nhắm tăng trải nghiệm của người mua đối với hệ thống.
    \item[(ii)] \textbf{Phát triển các chức năng quản lý chuyên nghiệp}: Tuy hệ thống đã có hệ thống con dành riêng cho nhà sách để quản lý, nhưng mới chỉ dừng lại ở mức quản lý cơ bản. Trong tương lai tôi sẽ phát triển thêm các chức năng quản lý nâng cao như tạo báo cáo hàng tháng, biểu đồ hiển thị doanh sô, v.v Các chức năng này sẽ nâng cao khả năng quản lý của nhà sách, giúp nhà sách có các quyết định và các chiến lược bán hàng tốt hơn.
    \item[(iii)] \textbf{Thiết kế lại giao diện phù hợp cho nhiều thiết bị}: Do hệ thống mới chỉ thiết kế giao diện cho các thiết bị như laptop, máy tính. Do đó các thiết bị nhỏ gọn như  smartphone sẽ chưa có được giao diện phù hợp, gây ảnh hưởng tới trải nghiệm của người dùng. Do đó trong tương lai, tôi sẽ thiết kế lại giao diện để phù hợp với nhiều loại thiết bị, cũng như phát triển đa nền tảng cho các thiết bị như android, ios.
    \item[(iv)] \textbf{tích hợp AI}: Trong tương lai, hệ thống sẽ tích hợp thêm từ các chức năng AI cơ bản như hệ gợi ý sách dựa theo sở thích của người dùng, chức năng tìm kiếm dựa theo các từ khóa đã được tìm kiếm trước đó, cho đến các hệ thống AI to hơn như Chatbot để tư vấn cho người dùng.  
\end{enumerate}

Thông qua việc phát triển thêm, cải tiến, khai phá các hướng đi mới trong tương lai, tôi tin rằng hệ thống của tôi sẽ ngày càng hoàn thiện và đáp ứng được đa dạng nhu cầu của người dùng trong lĩnh vực thương mại điện tử, ngoài ra có thể góp một phần nhỏ vào việc thúc đẩy sự phát triển của văn hóa đọc sách, xem sách trong xã hội.
\end{document}