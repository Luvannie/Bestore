\documentclass[../DoAn.tex]{subfiles}
\begin{document}

\begin{center}
    \Large{\textbf{TÓM TẮT NỘI DUNG ĐỒ ÁN}}\\
\end{center}
\vspace{1cm}
Với sự phát triển mạnh mẽ của thương mại điện tử trong những năm gần đây, nhu cầu của người dùng về việc tiếp cận và mua sắm những cuốn sách mong muốn ngày càng tăng. Tuy nhiên, các hệ thống bán sách truyền thống vẫn còn một số hạn chế như khó tiếp cận, chậm trong thanh toán và quản lý. Do đó, việc xây dựng một hệ thống bán sách trực tuyến đáp ứng được các nhu cầu của khách hàng cũng như nhà sách là rất cần thiết.

Để giải quyết vấn đề này, đề tài lựa chọn hướng tiếp cận phát triển một hệ thống bán sách trực tuyến với các tính năng như: quản lý danh mục sách, quản lý đơn hàng, thanh toán trực tuyến, kết nối giữa nhà sách và khách hàng. Hướng này được chọn vì nó có thể giải quyết được các hạn chế của hệ thống truyền thống, đáp ứng được các yêu cầu của khách hàng và nhà sách.

Hệ thống bán sách trực tuyến được thiết kế bao gồm các chức năng chính như: quản lý danh mục sách, quản lý đơn hàng, thanh toán trực tuyến, kết nối giữa nhà sách và khách hàng. Đây là những tính năng cần thiết để đáp ứng các yêu cầu của người mua và nhà bán sách, giúp cải thiện trải nghiệm mua sắm và quản lý hiệu quả hơn.

Đề tài đã thiết kế và xây dựng thành công hệ thống bán sách trực tuyến, giúp khách hàng dễ dàng tiếp cận và mua sắm các cuốn sách mong muốn, đồng thời cũng giúp các nhà sách kết nối với khách hàng tốt hơn, quản lý và thanh toán hiệu quả hơn. Hệ thống này sẽ góp phần thúc đẩy và nâng cao hoạt động kinh doanh trong lĩnh vực bán sách trực tuyến.
\begin{flushright}
Sinh viên thực hiện\\
\begin{tabular}{@{}c@{}}
\textit{(Ký và ghi rõ họ tên)}
\end{tabular}
\end{flushright}

\end{document}