\documentclass[../DoAn.tex]{subfiles}
\begin{document}

% Lưu ý: \textbf{Trước khi viết ĐATN, sinh viên cần đọc kỹ hướng dẫn và quy định chi tiết} về cách viết ĐATN trong Phụ lục A. Sinh viên tuân theo mẫu tài liệu này để viết báo cáo đồ án tốt nghiệp, vì tài liệu này đã được căn chỉnh, chỉnh sửa theo đúng chuẩn báo cáo kỹ thuật đồ án tốt nghiệp (ISO 7144:1986). Sinh viên viết trực tiếp vào file này, chỉ chỉnh sửa nội dung, và không viết trên file mới.

% \textbf{Khi đóng quyển ĐATN}, sinh viên cần lưu ý tuân thủ hướng dẫn ở phụ lục A.9

% \textbf{SV cần đặc biệt lưu ý cách hành văn}. Mỗi đoạn văn không được quá dài và cần có ý tứ rõ ràng, bao gồm duy nhất một ý chính và các ý phân tích bổ trợ để làm rõ hơn ý chính. Các câu văn trong đoạn phải đầy đủ chủ ngữ vị ngữ, cùng hướng đến chủ đề chung. Câu sau phải liên kết với câu trước, đoạn sau liên kết với đoạn trước. Trong văn phong khoa học, sinh viên không được dùng từ trong văn nói, không dùng các từ phóng đại, thái quá, các từ thiếu khách quan, thiên về cảm xúc, về quan điểm cá nhân như “tuyệt vời”, “cực hay”, “cực kỳ hữu ích”, v.v. Các câu văn cần được tối ưu hóa, đảm bảo rất khó để thể thêm hoặc bớt đi được dù chỉ một từ. Cách diễn đạt cần ngắn gọn, súc tích, không dài dòng.

% Mẫu ĐATN này được thiết kế phù hợp nhất với đa số các đề tài xây dựng phần mềm ứng dụng. Với các dạng đề tài khác (giải pháp, nghiên cứu, phần mềm đặc thù, v.v.), sinh viên dựa trên cấu trúc và hướng dẫn của báo cáo này để đề xuất và trao đổi với giáo viên hướng dẫn để thiết kế khung báo cáo đồ án cho phù hợp. Sinh viên lưu ý \textbf{trong mọi trường hợp, SV luôn phải sử dụng định dạng báo cáo này, và phải đọc kỹ toàn bộ các hướng dẫn từ đầu tới cuối.} Các hướng dẫn không chỉ áp dụng riêng cho đề tài ứng dụng, mà còn phù hợp với các dạng đề tài khác. Ngoài ra, trong mẫu ĐATN này đã được tích hợp một số hướng dẫn dành riêng cho đề tài nghiên cứu.

Chương 1 có độ dài từ 3 đến 6 trang với các nội dung sau đây

\section{Đặt vấn đề}
\label{section:1.1}
% Khi đặt vấn đề, sinh viên cần làm nổi bật mức độ cấp thiết, tầm quan trọng và/hoặc quy mô của bài toán của mình.%

% Gợi ý cách trình bày cho sinh viên: Xuất phát từ tình hình thực tế gì, dẫn đến vấn đề hoặc bài toán gì. Vấn đề hoặc bài toán đó, nếu được giải quyết, đem lại lợi ích gì, cho những ai, còn có thể được áp dụng vào các lĩnh vực khác nữa không. Sinh viên cần lưu ý phần này chỉ trình bày vấn đề, tuyệt đối không trình bày giải pháp.%

Trong thời đại công nghệ số hiện nay, việc mua bán sách trực tuyến đang trở nên ngày càng phổ biến. Ngành sách truyền thống phải đối mặt với nhiều thách thức, từ việc cạnh tranh với các trang bán sách trực tuyến đến việc duy trì lượng độc giả đến các cửa hàng truyền thống. Nhiều độc giả, đặc biệt là người trẻ, đã chuyển sang mua sắm sách trực tuyến do sự tiện lợi, đa dạng và khả năng so sánh giá cả. Tuy nhiên, các hệ thống bán sách trực tuyến hiện nay vẫn còn một số hạn chế như giao diện không thân thiện, khó tìm kiếm và khó đặt mua sách.

Việc phát triển một hệ thống bán sách trực tuyến hiệu quả và thân thiện với người dùng không chỉ giúp khắc phục những hạn chế hiện tại mà còn mở ra cơ hội thúc đẩy thói quen đọc sách trong cộng đồng. Hệ thống này nếu được triển khai thành công sẽ mang lại nhiều lợi ích, không chỉ cho người tiêu dùng, nhà xuất bản, và các nhà sách có quy mô vừa và nhỏ có cơ hội cạnh tranh với các chuỗi nhà sách lớn đã có hệ thống bán sách trực tuyến riêng, không cần phụ thuộc vào các trang thương mại điện tử, góp phần vào sự phát triển của ngành công nghiệp sách và. Hơn nữa, hệ thống này còn có thể được áp dụng và mở rộng ra các lĩnh vực bán lẻ trực tuyến khác, từ đó tạo điều kiện thuận lợi cho người tiêu dùng và góp phần vào sự phát triển kinh tế xã hội.\

Trên cơ sở những phân tích trên, đề tài "Hệ thống bán sách trực tuyến" được đề xuất nhằm góp phần giải quyết những vấn đề hiện tại của ngành sách truyền thống và đem lại trải nghiệm tốt hơn cho người tiêu dùng.

\section{Mục tiêu và phạm vi đề tài}
\label{section:1.2}
% Sinh viên trước tiên cần trình bày tổng quan các kết quả của các nghiên cứu hiện nay cho bài toán giới thiệu ở phần \ref{section:1.1} (đối với đề tài nghiên cứu), hoặc về các sản phẩm hiện tại/về nhu cầu của người dùng (đối với đề tài ứng dụng). Tiếp đến, sinh viên tiến hành so sánh và đánh giá tổng quan các sản phẩm/nghiên cứu này.

% Dựa trên các phân tích và đánh giá ở trên, sinh viên khái quát lại các hạn chế hiện tại đang gặp phải. Trên cơ sở đó, sinh viên sẽ hướng tới giải quyết vấn đề cụ thể gì, khắc phục hạn chế gì, phát triển phần mềm \textbf{có các chức năng chính gì}, tạo nên đột phá gì, v.v.

% Trong phần này, sinh viên lưu ý chỉ trình bày tổng quan, không đi vào chi tiết của vấn đề hoặc giải pháp. Nội dung chi tiết sẽ được trình bày trong các chương tiếp theo, đặc biệt là trong Chương 5.


Hiện nay, trên thị trường đã có nhiều hệ thống bán sách trực tuyến nhằm đáp ứng nhu cầu mua sắm của người tiêu dùng, như Tiki, Fahasa, và Vinabook. Các hệ thống này đều cung cấp một số tính năng cơ bản như tìm kiếm sách, đặt hàng, và giao hàng tận nơi. Tuy nhiên, chúng vẫn gặp phải một số hạn chế như giao diện người dùng phức tạp, chi phí phát triển và vận hành lớn. Dựa trên những phân tích và đánh giá này, đề tài của tôi hướng tới phát triển một hệ thống bán sách trực tuyến khắc phục các hạn chế hiện tại. Hệ thống sẽ tập trung vào việc cải thiện trải nghiệm người dùng thông qua giao diện thân thiện, đơn giản và dễ sử dụng, tối ưu chi phí phát triển và vận hành, phù hợp hơn với các nhà sách có quy mô vừa và nhỏ. Bên cạnh đó, hệ thống còn sẽ phát triển thêm các tính năng về quản lý để hỗ trợ cho nhà sách,các tính năng hỗ trợ tương tác giữa người dùng với nhau và với hệ thống, tạo nên một môi trường mua sắm trực tuyến tiện lợi và hiệu quả hơn. Qua đó, chúng tôi hy vọng sẽ đem lại một giải pháp đột phá, góp phần nâng cao chất lượng dịch vụ và thúc đẩy sự phát triển của ngành bán sách trực tuyến.

\section{Định hướng giải pháp}
\label{section:1.3}
% Từ việc xác định rõ nhiệm vụ cần giải quyết ở phần \ref{section:1.2}, sinh viên đề xuất định hướng giải pháp của mình theo trình tự sau: (i) Sinh viên trước tiên trình bày sẽ giải quyết vấn đề theo định hướng, phương pháp, thuật toán, kỹ thuật, hay công nghệ nào; Tiếp theo, (ii) sinh viên mô tả ngắn gọn giải pháp của mình là gì (khi đi theo định hướng/phương pháp nêu trên); và sau cùng, (iii) sinh viên trình bày đóng góp chính của đồ án là gì, kết quả đạt được là gì.

% Sinh viên lưu ý không giải thích hoặc phân tích chi tiết công nghệ/thuật toán trong phần này. Sinh viên chỉ cần nêu tên định hướng công nghệ/thuật toán, mô tả ngắn gọn trong một đến hai câu và giải thích nhanh lý do lựa chọn.

Để giải quyết các hạn chế của các hệ thống bán sách trực tuyến hiện có như giao diện người dùng phức tạp, chi phí phát triển và vận hành lớn, tôi định hướng giải pháp theo hai hướng chính:

(i) Sử dụng các công nghệ, kỹ thuật và phương pháp thiết kế giao diện người dùng hiện đại để xây dựng một giao diện thân thiện, đơn giản và dễ sử dụng. Cụ thể, tôi sẽ áp dụng các nguyên tắc thiết kế giao diện người dùng (UX) như tối ưu hóa trải nghiệm người dùng, tối giản hóa các tính năng, sử dụng các thành phần giao diện direct và intuitive. Bên cạnh đó, tôi cũng tận dụng các công nghệ mới như Angular để xây dựng một giao diện web responsive, tối ưu hóa hiệu năng và tốc độ tải trang.

(ii) Về mặt kiến trúc và công nghệ, tôi lựa chọn các công nghệ, kỹ thuật giúp tối ưu hóa chi phí phát triển và vận hành, phù hợp với các nhà sách quy mô vừa và nhỏ. Cụ thể, sử dụng Java,SpringBoot, MySQL để xây dựng backend nhằm giảm thiểu chi phí cơ sở hạ tầng. Bên cạnh đó, tôi còn phát triển các tính năng quản lý, tương tác người dùng để hỗ trợ nhà sách trong việc quản lý sản phẩm, đơn hàng và tương tác với khách hàng.

Thông qua việc áp dụng các định hướng và công nghệ trên, tôi hy vọng sẽ đem lại một giải pháp bán sách trực tuyến hiện đại, tiết kiệm chi phí, tối ưu trải nghiệm người dùng, đáp ứng tốt hơn nhu cầu của các nhà sách vừa và nhỏ. Đồng thời, hệ thống cũng sẽ cung cấp thêm các tính năng tương tác, quản lý để hỗ trợ doanh nghiệp trong kinh doanh online, góp phần thúc đẩy sự phát triển của ngành bán sách trực tuyến.

\section{Bố cục đồ án}
\label{section:1.4}
% Phần còn lại của báo cáo đồ án tốt nghiệp này được tổ chức như sau. 

% Chương 2 trình bày về v.v. 

% Trong Chương 3, em/tôi giới thiệu về v.v.

Phần còn lại của báo cáo đồ án tốt nghiệp này được tổ chức như sau:

Chương 2 trình bày về khảo sát hiện trạng và tổng quan chức năng của hệ thống bán sách trực tuyến. Nội dung chương này bao gồm việc khảo sát các hệ thống hiện có, biểu đồ usecase tổng quát và phân rã, quy trình nghiệp vụ, và đặc tả chức năng.

Trong Chương 3, tôi giới thiệu về các công nghệ sử dụng để phát triển hệ thống. 

Chương 4 mô tả chi tiết kết quả thực nghiệm, bao gồm thiết kế kiến trúc hệ thống (lựa chọn kiến trúc phần mềm, thiết kế tổng quan, và thiết kế chi tiết gói), thiết kế chi tiết (thiết kế giao diện, thiết kế lớp, và thiết kế cơ sở dữ liệu), quá trình xây dựng ứng dụng (thư viện và công cụ sử dụng, kết quả đạt được, minh họa các chức năng), kiểm thử và triển khai.

Chương 5 tập trung vào các giải pháp đóng góp của đồ án, nhằm cải thiện và hoàn thiện hệ thống bán sách trực tuyến.

Chương 6 đưa ra kết luận tổng quan về đề tài, những kết quả đạt được, cũng như các hướng phát triển trong tương lai.

Chương 7 cung cấp danh sách tài liệu tham khảo đã sử dụng trong quá trình nghiên cứu và phát triển đề tài.
% \textbf{Chú ý:} Sinh viên cần viết mô tả thành đoạn văn đầy đủ về nội dung chương. Tuyệt đối không viết ý hay gạch đầu dòng. Chương 1 không cần mô tả trong phần này. 

% Ví dụ tham khảo mô tả chương trong phần bố cục đồ án tốt nghiệp: Chương *** trình bày đóng góp chính của đồ án, đó là một nền tảng ABC cho phép khai phá và tích hợp nhiều nguồn dữ liệu, trong đó mỗi nguồn dữ liệu lại có định dạng đặc thù riêng. Nền tảng ABC được phát triển dựa trên khái niệm DEF, là các module ngữ nghĩa trợ giúp người dùng tìm kiếm, tích hợp và hiển thị trực quan dữ liệu theo mô hình cộng tác và mô hình phân tán.

% \textbf{Chú ý:} Trong phần nội dung chính, mỗi chương của đồ án nên có phần Tổng quan và Kết chương. Hai phần này đều có định dạng văn bản “Normal”, sinh viên không cần tạo định dạng riêng, ví dụ như không in đậm/in nghiêng, không đóng khung, v.v. 

% Trong phần Tổng quan của chương N, sinh viên nên có sự liên kết với chương N-1 rồi trình bày sơ qua lý do có mặt của chương N và sự cần thiết của chương này trong đồ án. Sau đó giới thiệu những vấn đề sẽ trình bày trong chương này là gì, trong các đề mục lớn nào.

% Ví dụ về phần Tổng quan: Chương 3 đã thảo luận về nguồn gốc ra đời, cơ sở lý thuyết và các nhiệm vụ chính của bài toán tích hợp dữ liệu. Chương 4 này sẽ trình bày chi tiết các công cụ tích hợp dữ liệu theo hướng tiếp cận “mashup”. Với mục đích và phạm vi của đề tài, sáu nhóm công cụ tích hợp dữ liệu chính được trình bày bao gồm: (i) nhóm công cụ ABC trong phần 4.1, (ii) nhóm công cụ DEF trong phần 4.2, nhóm công cụ GHK trong phần 4.3, v.v.

% Trong phần Kết chương, sinh viên đưa ra một số kết luận quan trọng của chương. Những vấn đề mở ra trong Tổng quan cần được tóm tắt lại nội dung và cách giải quyết/thực hiện như thế nào. Sinh viên lưu ý không viết Kết chương giống hệt Tổng quan. Sau khi đọc phần Kết chương, người đọc sẽ nắm được sơ bộ nội dung và giải pháp cho các vấn đề đã trình bày trong chương. Trong Kết chương, Sinh viên nên có thêm câu liên kết tới chương tiếp theo.

% Ví dụ về phần Kết chương: Chương này đã phân tích chi tiết sáu nhóm công cụ tích hợp dữ liệu. Nhóm công cụ ABC và DEF thích hợp với những bài toán tích hợp dữ liệu phạm vi nhỏ. Trong khi đó, nhóm công cụ GHK lại chứng tỏ thế mạnh của mình với những bài toán cần độ chính xác cao, v.v. Từ kết quả nghiên cứu và phân tích về sáu nhóm công cụ tích hợp dữ liệu này, tôi đã thực hiện phát triển phần mềm tự động bóc tách và tích hợp dữ liệu sử dụng nhóm công cụ GHK. Phần này được trình bày trong chương tiếp theo – Chương 5.

\end{document}